\documentclass[10pt]{article}
\usepackage[brazil]{babel}
\usepackage[utf8]{inputenc}
\usepackage[T1]{fontenc}
\usepackage{graphicx}

\title{Guia de funcionamento\\netstat\_snmp}
    
\author{Aluno: \textbf{Gustavo José Neves da Silva}}


\begin{document}
    
    \date{}
    
    \maketitle
    
    \newpage
    \tableofcontents 
    \newpage
    
    
\section{Requisitos} \label{requisitos}
Nesta seção, são apresentados os requisitos mínimos para a utilização do presente software.
	\subsection{requisitos:cliente}
		\begin{table}[h]
  			\centering
   			\begin{tabular}{| c | c |}     
      			\hline
      			\textbf{Browser} & \textbf{Versão} \\ \hline \hline
      			Chrome & 51 \\ \hline
	  			Firefox & 54 \\ \hline
	  			Edge & 14 \\ \hline
	  			Safari & 10 \\ \hline
	  			Opera & 38 \\ \hline
    		\end{tabular}
    		\caption{Versão mínima do browser}
    		\label{tabela:Browser}
		\end{table}
		
	\subsection{Servidor} \label{requisitos:servidor}
		\begin{table}[h]
   			\centering
   			\begin{tabular}{| c | c |}     
      			\hline
      			\textbf{Package} & \textbf{Versão} \\ \hline \hline
      			express & 4.17.1 \\ \hline
	  			net-snmp & 1.2.4 \\ \hline
    		\end{tabular}
    		\caption{Versão mínima dos packages}
    		\label{tabela:Packages}
		\end{table}
		
		\begin{table}[h]
   			\centering
   			\begin{tabular}{| c | c |}     
      			\hline
      			\textbf{Interpretador} & \textbf{Versão} \\ \hline \hline
      			Node.js & 12.13.0 \\ \hline
    		\end{tabular}
    		\caption{Versão mínima do interpretador}
    		\label{tabela:Node}
		\end{table}
		
\section{Utilização}
	\subsection{Cenários de uso} \label{cenario}
%\section{Cenários de uso} \label{cenario}
		O presente software oferece suporte a todos os sistemas operacionais, seja no cliente ou no servidor, desde que sejam atendidos os requisitos apresentados na Seção~\ref{requisitos}.
	
	\subsection{Cliente} \label{utilizacao:cliente}
	
	A Figura~\ref{fig:telaCliente}, ilustra a visão que o cliente possui do software. Para realizar um busca devem ser realizados os seguintes passos:
	
%	\begin{itemize}
%		\item Inserir o ip do host que se deseja consultar~(Figura~\ref{fig:telaCliente}, item 1)
%		\item Inserir a community~(Figura~\ref{fig:telaCliente}, item 2)
%		\item Selecionar o(s) protocolo(s)~(Figura~\ref{fig:telaCliente}, item(s) 3 e/ou 4)
%		\item Clicar no botão Consultar~(Figura~\ref{fig:telaCliente}, item 5)
%	\end{itemize}
	\begin{itemize}
		\item Inserir o endereço ip do servidor seguido de ":8081"\footnote{A porta de recebimento do servidor pode ser alterada no arquivo fonte.}~(Figura~\ref{fig:telaCliente}, item 1)
		\item Inserir o endereço ip do host que se deseja consultar~(Figura~\ref{fig:telaCliente}, item 2)
		\item Inserir a community~(Figura~\ref{fig:telaCliente}, item 3)
		\item Selecionar o(s) protocolo(s)~(Figura~\ref{fig:telaCliente}, item(s) 4 e/ou 5)
		\item Clicar no botão Consultar~(Figura~\ref{fig:telaCliente}, item 6)
	\end{itemize}
	
	\begin{figure}[htbp]
%    \centerline{\includegraphics[width=0.8\textwidth]{imagens/telaCliente.png}}
%	\centerline{\includegraphics[width=\textwidth]{imagens/telaCliente.png}}
	\centerline{\includegraphics[width=1.5\textwidth]{imagens/tela2.png}}
    \caption{Tela principal do cliente}
    \label{fig:telaCliente}
   	\end{figure}
	
	\subsection{Servidor} \label{utilizacao:servidor}
	A Figura~\ref{fig:telaCliente}, ilustra a visão do servidor do software. Para iniciar o servidor é necessário realizar apenas o sequinte passo: \verb+node diretorioDeTrabalho/index.js+, como ilustrado pela Figura~(Figura~\ref{fig:telaServidor}, item 1).
	
	\begin{figure}[htbp]
	\centerline{\includegraphics[width=1.5\textwidth]{imagens/telaServidor.png}}
    \caption{Tela principal do servidor}
    \label{fig:telaServidor}
   	\end{figure}

\end{document}